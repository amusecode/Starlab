
% Binaries.tex:  Some background on how kira deals with binary motion.
%		 Steve McMillan, August 1998

\documentclass{article}

\textwidth 6.5in
\textheight 9in
\voffset -0.9in
\hoffset -1in

\pagestyle{empty}

\newcommand{\mbar}{\langle m \rangle}
\newcommand{\simlt}{\ \hbox{\rlap{\raise0.425ex\hbox{$<$}}\lower0.65ex\hbox{$\sim$}}\ }
\newcommand{\simgt}{\ \hbox{\rlap{\raise0.425ex\hbox{$>$}}\lower0.65ex\hbox{$\sim$}}\ }

\begin{document}

\begin{center}
	\LARGE{\bf Binary Handling in {\tt kira}}\\
\vskip 6pt
	\Large{\it Steve McMillan}\\
	{\it August 1998}
\end{center}

\bigskip
\noindent
{\it Notation:} A binary has mass $m_b$, separation $r$, semi-major
axis $a$, and period $P$.  We define $R_b = \max(r,a)$.  A perturber
has mass $m_p$ and distance $D_p$.  We use standard N-body units, so
the cluster mass, length, and time scales are all 1.  The mean stellar
mass is therefore $\mbar \equiv 1/N$.

\bigskip\bigskip\noindent
{\large\it 1.~ Binaries without unperturbed motion}

\bigskip\noindent Binaries are constrained by the following
operational parameters:

\begin{enumerate}

\item	A close pair (of total mass $m_b$) is treated as a ``binary''
	(i.e. motion of the center of mass is separated from the
	relative motion of the components) when 
$$
	r < r_{crit} \equiv \frac{f}{N} {\cal M}_1^\alpha\,,
$$
	where $f$ is a free parameter, ${\cal M}_1 = m_b/2\mbar$, and
	$\alpha =$ 0, 0.5, or 1, depending on the precise criterion
	(distance, force, or potential, respectively) used for binary
	recognition.  A binary is split back into single particles
	when $r > 2.5\, r_{crit}$.

\item	The dimensionless number $\gamma$ controls which neighbors are
	regarded as perturbers of a given binary.  Particle $p$ is a
	perturber of binary $b$ if
$$
	\frac{{\cal M}_2 R_b}{D_p^3} \ge \frac{\gamma}{R_b^2}\,,
$$
	or
$$
	D_p < \gamma^{-1/3} {\cal M}_2^{1/3} R_b\,,
$$
	where ${\cal M}_2 = \max(2m_p/m_b, 1)$.  The use of ${\cal
	M}_2$ allows us to include all perturbers within a given
	distance, as well as more massive particles at greater
	distances.

\item	The number of perturbers $N_p$ must be less than $N_{p,max}$,
	the maximum allowed length of the perturber list in the {\tt
	hdyn} class.

\item	The speed-up obtained by using the GRAPE hardware is a factor
	of $S_G \sim 100$--$500$.

\end{enumerate}

The value of $\gamma$ is determined primarily by the accuracy desired,
as the error incurred in binary integration scales as $\gamma$.
Empirically, we find that a typical error per unit time associated
with perturbed binary motion is a few times $\gamma$ (in standard
units, with no initial binaries).  We take $\gamma\simlt 10^{-7}$
(with accuracy parameter $\eta = 0.1$) for reasonable accuracy.  Once
$\gamma$ is chosen, the choice of $f$ is dictated by by efficiency
considerations and the values of the other parameters.  The number of
perturbers is
$$
	N_p \sim n D_p^3,
$$
where $n$ is the local density.  We define $n = \rho N$, where $\rho$
is an overdensity factor, which could easily lie in the range
100--1000 for binary interactions in the core.  Thus, the limit on the
total number of perturbers for a binary of size $r_{crit}$ requires
$$
	f \simlt \biggl(\frac{N^2 N_{p,max} \gamma}
			{\rho {\cal M}_1^{3\alpha} {\cal M}_2}\biggr)^{1/3}\,.
$$
For $N = 10^4N_4$, $\gamma = 10^{-7}\gamma_7$, and $N_{p,max} = 256$,
and taking $\rho = {\cal M}_1 = {\cal M}_2 = 1$, this implies
$$
	f \simlt 14 N_4^{2/3} \gamma_7^{1/3}
$$
to avoid perturber-list overflow.  Taking the factors in the
denominator into account, the ``standard'' choice of $f = 1$ seems
conservatively safe.

For a binary with $N_p$ perturbers, we must perform $2 N_p$ pairwise
force calculations per time step.  With $\sim$200 steps per orbit and
an orbital period of $2\pi\sqrt{a^3/m_b}$, this implies a total of
$$
	N_{f,bin} \sim 100 N^{1/2} a^{3/2} \gamma^{-1}
	 		\rho {\cal M}_1^{1/2} {\cal M}_2
$$
force calculations per orbit (replacing $R_b$ by $a$ in the definition
of $D$ above, and taking $200\sqrt{2}/\pi \sim 100$).  For the same
binary treated as two separate particles, the equivalent number of
force calculations is obtained by replacing $N_p$ by $N$ and dividing
by $S_G$:
$$
	N_{f,nobin} \sim \frac{100}{S_G} N^{1/2} a^{-3/2} {\cal M}_1^{1/2}\,.
$$
We thus do less work in the binary case when
$$
	a^3 \gamma^{-1} \rho {\cal M}_2 S_G \simlt 1\,,
$$
or, substituting $a = r_{crit}$,
$$
	f \simlt \frac{\gamma^{1/3} N}
		      {{\cal M}_1^{\alpha} {\cal M}_2^{1/3}
			\rho^{1/3}S_G^{1/3}}\,.
$$
Again neglecting the mass and density factors (possibly amounting to a
factor of 10 or more, note), and taking $S_G \sim 100$, we obtain
$$
	f \simlt 10 \gamma_7^{1/3} N_4\,.
$$
Given the extra factors in practice, and given the fact that the
direct calculation is likely to be more accurate than the perturbed
binary formulation, even with $\gamma = 10^{-7}$, the standard choice
of $f \sim 1$ again seems reasonable.  In the core of a collapsed
system with a wide range of masses, a reduction of $f$ to $\sim
0.25$--0.5 may be desirable.

The original implementation of perturber lists associated the list
with the top-level node of a binary/multiple system and used the same
list for all internal motion.  The list is recomputed at each
center-of-mass time step.  As of August 1998, we optionally allow the
use of low-level perturber lists for more efficient treatment of
multiple systems.  As with the top-level lists, low-level lists are
updated each center-of-mass time step; membership is drawn from the
top-level list, applying the same perturbation criteria for inner
binaries as is used in creating the top-level list for the outer
binary.

\bigskip\bigskip\noindent
{\large\it 2.~ Unperturbed binaries}

\bigskip\noindent Direct integration of a lightly perturbed binary in
an orbit that is not too eccentric, with accuracy parameter $\eta =
0.1$ and $\sim 300$ steps per orbit, conserves energy at the level of
$\sim10^{-8}$ (relative error) per orbit.  (The number of steps is
modified over and above the standard Aarseth criterion, depending on
the mass of the binary, in order to increase the integration accuracy
for energetic binaries.)  While this is a small error, it is
systematic, always {\it decreasing} the total energy of the system.
Two types of unperturbed binary motion are recognized, defined by
additional parameters:

\begin{enumerate}

\item	{\it Pericenter reflection.}  A binary may be regarded as
	unperturbed during pericenter passage (i.e. its motion is
	treated as simple two-body motion and the binary step amounts
	to a reflection about the pericenter point) if its
	perturbation is less than $\gamma_p$.

\item	{\it Fully unperturbed motion.}  If the external perturbation
	(determined at apocenter) is less than $\gamma_f$, the entire
	binary orbit is treated as unperturbed, allowing time steps
	many orbits long.

\end{enumerate}

During unperturbed motion, the internal motion of the components is
treated in the kepler approximation, and the binary is followed as a
point mass.  The component positions and velocities are taken into
account in computing the total energy of the system, however, and are
updated at the end of each unperturbed step.  A tidal error is incurred
during pericenter reflection, as the configuration of the binary
relative to the external field changes during the step with no
corresponding change in acc or jerk until the end of the unperturbed
motion.  At present, this error is simply included in the overall
integration error.  It should probably be accumulated as a separate
item in {\tt kira\_counters}.  This tidal error is seen immediately in
the instantaneously energy, and may also introduce an anomalously
large jerk into the motions of the binary center of mass and the
closest perturbers.

In order to minimize the tidal error in pericenter reflection (which
often is systematic, as the same reflection configuration is typically
repeated many times), $\gamma_p$ is kept small---$0.1\gamma$ is the
standard choice.  Systematic tidal errors are less of a problem for
fully unperturbed motion, which proceeds an orbit at a time.
Moreover, most perturbative terms are periodic in nature, and so
vanish over a full orbital period, allowing $\gamma_f$ to be
substantially greater than $\gamma$.  We adopt $\gamma_f = 100
\gamma$.

As a practical matter, long-lived multiple systems, such as
hierarchical triples, can pose severe problems, as the inner binary is
often not ``unperturbed'' by the above criteria, but the system
survives for many binary periods, resulting in large energy errors.  A
triple system is treated as unperturbed (i.e. both the inner and outer
orbits are followed as two-body motion, and the center of mass is
advanced as a point mass) if

\begin{enumerate}

\item	The outer orbit is ``weakly perturbed''---that is, its
	dimensionless perturbation is less than some value
	$\gamma_t$---and

\item	The internal motion is stable---that is, the ratio of the outer
	periastron to the inner semi-major axis exceeds some critical
	ratio $R$.

\end{enumerate}

We currently (pragmatically) take $\gamma_t = 10^{-4}$ and $R = 5$.
To minimize the tidal error, both the inner and outer orbits are
advanced through integral numbers of periods.  Any residual error is
absorbed in the same way as for binary motion.

Unperturbed binary--binary systems are treated similarly, applying the
stability criterion to each interior binary separately.  Hierarchical
systems are treated as unperturbed if the inner system is stable (by
the above definition) and the outer orbit is weakly perturbed.

\end{document}
